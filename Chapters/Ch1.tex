\chapter{Sample Chapter}
\lipsum[4]

\section{Sample Equations}
\lipsum[4] 

\subsection{Greek Letters}
\[ \alpha,  \beta,  \gamma, \Gamma, \pi, \Pi, \phi, \varphi, \mu, \Phi, \xi, \zeta \]
\[ \cos(2\theta\phi) = \cos^2 \theta\phi - \sin^2 \theta\phi \]
\subsection{Delimiters}
There are many types of delimiters one can use:
\[ ( a ), [ b ], \{ c \}, | d |, \| e \|,
\langle f \rangle, \lfloor g \rfloor,
\lceil h \rceil, \ulcorner i \urcorner \]
See how the delimiters are of reasonable size in these examples
\[
        \left(a+b\right)\left[1-\frac{b}{a+b}\right]=a\,,
\]
\[
        \sqrt{|xy|}\leq\left|\frac{x+y}{2}\right|,
\]
even when there is no matching delimiter
\[
        \int_a^bu\frac{d^2v}{dx^2}\,dx
        =\left.u\frac{dv}{dx}\right|_a^b
        -\int_a^b\frac{du}{dx}\frac{dv}{dx}\,dx.
\]
whereas vector problems often lead to statements such as
\[
        u=\frac{-y}{x^2+y^2}\,,\quad
        v=\frac{x}{x^2+y^2}\,,\quad\text{and}\quad
        w=0\,.
\]
\subsection{Multiple Fractions}
Typesetting continued fractions is easy:
\[
x = a_0 + \frac{1}{a_1 + \frac{1}{a_2 + \frac{1}{a_3 + a_4}}}
\]
However, as the fractions continue, they get smaller. If you want to keep the size consistent, use the display style; e.g.
\[
  x = a_0 + \frac{1}{\displaystyle a_1
          + \frac{1}{\displaystyle a_2
          + \frac{1}{\displaystyle a_3 + a_4}}}
\]
\subsection{Arrays}
Arrays of mathematics are typeset using one of the matrix environments as
in
\[
        \begin{bmatrix}
                1 & x & 0 \\
                0 & 1 & -1
        \end{bmatrix}\begin{bmatrix}
                1  \\
                y  \\
                1
        \end{bmatrix}
        =\begin{bmatrix}
                1+xy  \\
                y-1
        \end{bmatrix}.
\]
\[   
\begin{pmatrix}
2 & 3 & 4\\
5 & 6 & 7\\
8 & 9 & 10 \end{pmatrix} v = 0 \]
Case statements use cases:
\[
        |x|=\begin{cases}
                x, & \text{if }x\geq 0\,,  \\
                -x, & \text{if }x< 0\,.
        \end{cases}
\]
Many arrays have lots of dots all over the place as in
\[
        \begin{matrix}
                -2 & 1 & 0 & 0 & \cdots & 0  \\
                1 & -2 & 1 & 0 & \cdots & 0  \\
                0 & 1 & -2 & 1 & \cdots & 0  \\
                0 & 0 & 1 & -2 & \ddots & \vdots \\
                \vdots & \vdots & \vdots & \ddots & \ddots & 1  \\
                0 & 0 & 0 & \cdots & 1 & -2
        \end{matrix}
\]

\subsection{Accents}
Mathematical accents are performed by a short command with one
argument, such as
\[
        \tilde f(\omega)=\frac{1}{2\pi}
        \int_{-\infty}^\infty f(x)e^{-i\omega x}\,dx\,,
\]
or
\[
        \dot{\vec \omega}=\vec r\times\vec I\,.
\]
\subsection{Multiline equations and aligned environments}
New lines (\\ ) do not work in equation environments. To achieve alignment of equations, use the aligned  package to produce multiline aligned math, such as:
\newline
\begin{center}
\begin{align}
F ={} & \{F_{x} \in  F_{c} : (|S| > |C|) \\
      & \cap (\mathrm{minPixels}  < |S| < \mathrm{maxPixels}) \\
      & \cap (|S_{\mathrm{conected}}| > |S| - \epsilon) \}
\end{align}
\end{center}
and also:
\begin{center}
\begin{align}
A_0 & =   \frac{1}{(\alpha+t_x)^{r+s+x}}{}_2 F_1\left( r+s+x,x+1;r+s+x+1;\frac{\alpha-\beta}{\alpha + t_x} \right) \\
& \quad - \frac{1}{(\alpha+T)^{r+s+x}}{}_2 F_1\left( r+s+x,x+1;r+s+x+1;\frac{\alpha-\beta}{\alpha + T} \right),
\end{align}
\end{center}

\begin{theorem}
For any nonnegative integer n, we have
$(1+x)^n = \sum_{i=0}^n {n \choose i} x^i$
\end{theorem}

